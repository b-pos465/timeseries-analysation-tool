\documentclass[12pt]{article}

\usepackage[ngerman]{babel}

\usepackage[utf8]{inputenc}

\title{Bachelorarbeit Informatik}
\author{Stefan Bechert}
\date{\today}



\begin{document}

\maketitle
\newpage
\tableofcontents
\newpage

\section{Einleitung}
	\subsection{Motivation}
		Zeitreihen, 
	
		Wofuer braucht man diese Arbeit? Beispiele aus Industrie.
		\begin{enumerate}
			\item{Einsatz in vielen industriellen Projekten.}
			\item{Fehleranalyse.}
			\item{Vergleiche in flexiblen zeitlichen Dimensionen.}
		\end{enumerate}
	\subsection{Zielsetzung}
		Welche Ziele sollen erreicht werden? 
	\subsection{Gliederung}
		Wie ist die Arbeit aufgebaut? Warum ist sie das?
		
\section{Grundlagen}
		Welche theoretischen und praktischen Grundlagen sind zur Umsetzung dieser Arbeit erforderlich?
	\subsection{Plotly}
		Wieso Plotly? Grundlagen?
	\subsection{Zeitreihen, Repräsentationen und Operationen}
		\subsubsection{Definition}
		\label{sec: def}
		Eine \textbf{äquidistante Zeitreihe} ist ein Tripel 
			\begin{equation}
				D^{2} = (V, i, s)
			\end{equation}
		wobei
			\begin{equation}
				V = [v_{0}, v_{1}, ..., v_{n - 1}]
			\end{equation}
		ein Array aus n Kennwerten darstellt.
		$i$ beschreibt den zeitlichen Abstand zwischen zwei Kennwerten in Millisekunden und s repräsentiert den Starzeitpunkt der Zeitreihe in 		Millisekunden seit dem 01.01.1970.
		\subsubsection{Repräsentationen}
		Die Definition in \ref{sec: def} gilt im folgenden als die initiale Repräsentation einer Zeitreihe in der Anwendung. (Passt hier eig 			nicht, sollte eher zu Implementierung)
		
			\paragraph{2D - Repräsentation}
				Die Repräsentation nach \ref{sec: def} ist eine 2-dimensionale Darstellung der Kennzahlen, da nur ein Index jede Kennzahl 						eindeutig referenziert.
				
			\paragraph{3D - Repräsentation}
				Jede Zeitreihe kann auch in Abhängigkeit von zwei unterschiedlichen Zeitdimensionen angegeben werden. \\[0,3cm]
				Sei $D^{2} = (V, i, s)$ eine Zeitreihe in 2D-Repräsentation. So lässt sich diese Zeitreihe auch darstellen in 3D-Repräsentation als
				\begin{equation}
					D^{3} = (V', i, s)
				\end{equation}
				mit
				\begin{equation}
					V' = [[v_{0}, v_{1}, ..., v_{t_{1} - 1}]_{0},[v_{0}, v_{1}, ..., v_{t_{2} - 1}]_{1},...,[v_{0}, v_{1}, ..., v_{t_{m} - 1}]_{m}]
				\end{equation}
				wobei 
				\begin{equation}
					\sum_{k=1}^{m}t_{k} = |V|.
				\end{equation}
				Falls
				\begin{equation}
					t_{1}=t_{2}=...=t_{m}
				\end{equation}
				gilt, so nennt man $V'$ eine \textbf{homogene}, ansonsten eine \textbf{inhomegene} 3D-Repräsentation von D.
			\paragraph{Folgerung Homogen 3D}
				Sei $D^{3}$ als homogen gegeben. So folgt:
				\begin{equation}
					\sum_{k=1}^{m}t_{k} = t*m = |V|.
				\end{equation}
				
			\paragraph{nD - Repräsentation, Ueberlegung}
				Allgemein kann man jede Zeitreihe in $n$ vielen Dimensionen darstellen (da beliebig kleine Zeitschritte), allerdings erreicht man dadurch irgendwann keine neuen Informationen mehr sondern ausschliesslich Kennwertduplizierung.
		\subsubsection{Operationen - allgemein}
			Sei $K^{n}$ die Menge aller Zeitreihen in $n$D-Repraesentation.		
		
			\paragraph{Unterteilung}
				$\forall n \geq 2$ gilt:\\
				Sei $D^{n} = (V, i, s)$ eine Zeitreihe in nD-Repraesentation.\\[0.3cm]
				Eine Unterteilung $\delta_{n}$ ist eine Abbildung 
				\begin{equation}
					\delta_{n}: K^{n} \rightarrow K^{n+1}.
				\end{equation}
				
			\paragraph{Aggregation}
				$\forall n \geq 3$ gilt:\\
				Sei $D^{n} = (V, i, s)$ eine Zeitreihe in nD-Repraesentation.\\[0.3cm]
				Eine Aggregation $\phi_{n}$ ist eine Abbildung 
				\begin{equation}
					\phi_{n}: K^{n} \rightarrow K^{n-1}.
				\end{equation}
				
			\begin{picture}(100,200)
				\put(0,140) {\line(1,0){180}}			
			
				% erste Spalte - Unterteilung
				\put(30,0) {$D^{n}$}
				\put(40,25) {\vector(0,-1){10}}
				\put(30,30) {$D^{n-1}$}
				\put(40,55) {\vector(0,-1){10}}
				\put(35,60) {...}
				\put(40,85) {\vector(0,-1){10}}
				\put(30,90) {$D^{3}$}
				\put(40,115) {\vector(0,-1){10}}
				\put(30,120) {$D^{2}$}
				\put(0,150) {\textbf{Unterteilung}}
				
				% zweite Spalte - Aggregation
				\put(130,0) {$D^{n}$}
				\put(140,15) {\vector(0, 1){10}}
				\put(130,30) {$D^{n-1}$}
				\put(140,45) {\vector(0, 1){10}}
				\put(135,60) {...}
				\put(140,75) {\vector(0, 1){10}}
				\put(130,90) {$D^{3}$}
				\put(140,105) {\vector(0, 1){10}}
				\put(130,120) {$D^{2}$}
				\put(100,150) {\textbf{Aggregation}}	
			\end{picture}	
		
	\subsection{Pivot Tabellen}
		Sei $D = (V, i, s)$ eine Zeitreihe. So kann $D$ auch als Tabelle dargestellt werden:\\[0,4cm]
			\begin{tabular}{l|l|l|l|l|l}
				Zeit in msec & $s$     & $s + i$ & $s + 2*i$ & $...$ & $s+(n-1)*i$\\
					\hline
				Kennwert & $v_{0}$ & $v_{1}$ & $v_{2}$   & $...$ & $v_{n-1}$\\
			\end{tabular}
\section{Anforderungsanalyse und Konzept}
	Was soll die Software am Ende koennen? Wie sollte man rangehen?
\section{Umsetzung}
	Tatsaechliche Umsetzung: Probleme, Entscheidungen, 
	
				\paragraph{Wichtige Forderungen TODO aber merrken}
				Dass die Teilarrays in der 3D Repraesentation die gleich Groesse haben ist fuer den spaeteren Render-Prozess von hoher Wichtigkeit, da andernfalls Daten in der Darstellung verloren gehen. Nichts destotrotz kann eine Zeitreihe auch in ungleiche Teile zerteilt werden (z.B. Monate), die allerdings zur Darstellung auf jeden Fall aggregiert werden muessen um die Forderung einzuhalten.
\section{Zusammenfassung und Evaluation}
	\subsection{Auswertung}
		Wurden die Ziele erfuellt? Wenn ja zeigen. Wenn nein, wieso nicht?
		Zeitpunkt: Nach der Fertigstellung der software.
	\subsection{Ausblick}
		Was koennte man aufbauend auf dieser Arbeit noch machen?
\end{document}
